\documentclass[a4paper,11pt]{ltjsarticle}


% 数式
\usepackage{amsmath,amsfonts}
\usepackage{bm}
% 画像
\usepackage{graphicx}

\usepackage{caption}
\captionsetup{labelformat=empty,labelsep=none}
\begin{document}

\title{通信工学 演習レポート}
\author{氏名:関川 謙人\\学籍番号:2022531033}
\date{\today}
\maketitle

\section*{問1}
$11 \ 10 \ 00 \ 10 \ 11 \ 10$に最も近いグラフは(b)である。
\section*{問2}
(1)\ $p_{A}=p_{B}=p_{C}=p_{D}= 0.25 $である。このことから
\begin{align*}
  I(p_{A}) &= I(p_{B}) = I(p_{C}) = I(p_{D}) \\ 
  &= -\log_{2}\frac{1}{4} = 2
\end{align*}
エントロピー$H(X)$は
\begin{align*}
  H(X) &= -(4(0.25・\log_{2}\frac{1}{4})) \\
       &= 2
\end{align*}
(2)
$p_{A}= 0.5 \ p_{B}=p_{C}=0.125 \ p_{D}= 0.25$であるので
\begin{align*}
  I(p_{A}) &= -\log_{2}\frac{1}{2} = 1 \\
  I(p_{B}) &= I(p_{C}) = -\log_{2}\frac{1}{8} = 3 \\
  I(p_{D}) &= -\log_{2}\frac{1}{4} = 2
\end{align*}
以上より、エントロピー$H(X)$は
\begin{align*}
  H(X) &= -((0.5・\log_{2}\frac{1}{2})+(0.25・\log_{2}\frac{1}{4})+2(0.125・\log_{2}\frac{1}{8})) \\
       &= 1.75
\end{align*}
\section*{問3}
(1)
各アルファベットの出現頻度から以下の二分木を構成した
\begin{figure}[h]
\begin{center}
\includegraphics*[width = 8cm]{tuushin1.png}
\end{center}
\end{figure}

(2)符号語と符号長を組み合わせた表は以下の通りである

\begin{table*}[h]
  \centering
  \begin{tabular}{|c|c|c|c|} \hline
  文字 & 頻度 & 符号語 & 符号長 \\ \hline \hline
  A & 8 & 0 & 1 \\ \hline
  B & 4 & 10 & 2 \\ \hline
  C & 1 & 1110 & 4 \\ \hline
  D & 1 & 1111 & 4 \\ \hline
  E & 2 & 110 & 3 \\ \hline
  \end{tabular}
\end{table*}

(3)固定符号長であるため、平均は各々の符号長に一致する。よって$l_{1}=3$

ハフマン符号の平均符号長は
\begin{align*}
  l_{2} &= \frac{8+8+8+6}{16} = \frac{15}{8} \\
        &= 3.75
\end{align*}

このことからハフマン符号のエントロピー$H$は、
\begin{align*}
  H &= -((\frac{1}{2}・\log_{2}\frac{1}{2}) + 
  (\frac{1}{4}・\log_{2}\frac{1}{4}) + 
  (\frac{1}{8}・\log_{2}\frac{1}{8}) + 
  2・(\frac{1}{16}・\log_{2}\frac{1}{16})) \\
  &= 3.75
\end{align*}
以上のことからハフマン符号のエントロピーと平均符号長は等しいことがわかる。
\section*{問4}
(1)表にすると以下のようになる
\begin{table*}[h]
  \begin{minipage}[t]{.45\textwidth}
    \begin{center}
      \begin{tabular}{c|ccccc}
        & 1 & 2 & 3 & 4 & 5 \\ \hline
        $x_{1}$ & 0 & 1 & 1 & 1 & 1 \\
        $x_{2}$ & 1 & 0 & 0 & 1 & 0 \\
        $x_{3}$ & 1 & 0 & 1 & 1 & 1 \\
        $x_{4}$ & 0 & 1 & 0 & 1 & 0 \\
      \end{tabular}
      \caption{偶数パリティ検査符号}
    \end{center}
  \end{minipage}
  %
  \hfill
  %
  \begin{minipage}[t]{.45\textwidth}
    \begin{center}
      \begin{tabular}{c|ccccc}
        & 1 & 2 & 3 & 4 & 5 \\ \hline
        $x_{1}$ & 0 & 0 & 0 & 1 & 0 \\
        $x_{2}$ & 1 & 0 & 0 & 1 & 1 \\
        $x_{3}$ & 1 & 0 & 1 & 1 & 0 \\
        $x_{4}$ & 0 & 1 & 0 & 1 & 1 \\
      \end{tabular}
      \caption{奇数パリティ検査符号}
    \end{center}
  \end{minipage}
  \caption{(1)パリティ検査符号}
\end{table*}

(2)
パリティ検査符号について、ハミング重みと誤り検出の可否は以下のようになっている。
\begin{table*}[h]
  \begin{minipage}[t]{.45\textwidth}
    \begin{center}
      \begin{tabular}{c|c|c}
        信号 & ハミング重み$w_{H}(y)$ & 誤り検出 \\ \hline
        $y_{1}$ & 1 & 可 \\
        $y_{2}$ & 3 & 可 \\
        $y_{3}$ & 4 & 不可 \\
        $y_{4}$ & 3 & 可 \\
      \end{tabular}
    \end{center}
    \caption{偶数パリティ検査符号}
  \end{minipage}
  %
  \hfill
  %
  \begin{minipage}[t]{.45\textwidth}
    \begin{center}
      \begin{tabular}{c|c|c}
        信号 & ハミング重み$w_{H}(y)$ & 誤り検出 \\ \hline
        $y_{1}$ & 2 & 可 \\
        $y_{2}$ & 2 & 可 \\
        $y_{3}$ & 3 & 不可 \\
        $y_{4}$ & 4 & 可 \\
      \end{tabular}
    \end{center}
    \caption{奇数パリティ検査符号}
  \end{minipage}
  \caption{(2)パリティ検査符号のハミング重みと誤り検出の可否}
\end{table*}

(3)奇数パリティビットではハミング重みが奇数の時0、偶数の時1を返す。
$y_{5}$の4列目の要素がこの法則に反するため誤りは4列目にある。
また$y_{3}$の5列目がこの法則に反している。
以上のことから誤り箇所は$y_{3}$の4番目の要素といえる。
\end{document}