\documentclass[a4paper,12pt]{ltjsarticle}
\usepackage[margin=1in]{geometry}
\usepackage{amsmath}
\usepackage{amssymb}
\usepackage{array}
\usepackage{booktabs}

\title{ネットワーク設計と性能解析}
\author{}
\date{}

\begin{document}

\maketitle

\section{ネットワーク設計の基本}

\subsection*{性能とコストのバランス}
\begin{itemize}
    \item 資源増加の利点: 待ち時間やデータ損失の減少。
    \item 資源増加の欠点: コストの増加。
    \item 資源減少の利点: コスト削減。
    \item 資源減少の欠点: 待ち時間やデータ損失の増加。
\end{itemize}
\textbf{経済性を維持しつつ、通信品質を一定に保つことが求められる。}

\subsection*{トラヒック変動への対応}
\begin{itemize}
    \item トラヒック量が資源量を超えると輻輳(混雑)発生。
    \item 資源量がトラヒック量を超えると輻輳回避。
    \item 長期的に輻輳が継続しないよう設計する必要がある。
\end{itemize}

\section{待ち行列理論}

\subsection*{目的}
トラヒック量とネットワーク資源の関係から、システムの性能を定量的に評価する。

\subsection*{モデル化要素}
\begin{itemize}
    \item 到着分布: パケットが到着する確率分布(例: ポアソン分布)。
    \item サービス時間分布: 各パケットが処理される時間。
    \item サーバ数: パケットを処理する窓口数。
    \item バッファサイズ: 待ち行列の最大長。
\end{itemize}

\section{ポアソン分布}

\subsection*{特性}
\begin{itemize}
    \item 独立かつランダムな事象の発生を表す分布。
    \item 例: 単位時間あたりの呼(ネットワーク接続要求)の発生数。
\end{itemize}

\subsection*{条件}
\begin{itemize}
    \item 定常性: 発生率は時間に依存しない。
    \item 独立性: 発生する事象は互いに独立。
    \item 希少性: 微小時間内に複数の事象が発生する確率は無視できる。
\end{itemize}

\subsection*{確率式}
\[
P_k(t) = \frac{(\lambda t)^k e^{-\lambda t}}{k!}
\]
\begin{itemize}
    \item \( P_k(t) \): 時間 \( t \) に \( k \) 回発生する確率。
    \item \( \lambda \): 呼の発生率。
\end{itemize}

\section{スループット解析}

\subsection*{スループット (\( S \))}
\begin{itemize}
    \item 単位時間あたりの成功伝送データ量。
    \item 物理伝送速度で正規化され、\( 0 \leq S \leq 1 \)。
    \item トラヒック量 \( G \) とスループット \( S \) の関係で評価。
\end{itemize}

\subsection*{純アロハ方式の解析}
\begin{itemize}
    \item 衝突と再送: 衝突発生時にランダム時間後に再送。
    \item スループット公式:
    \[
    S = G e^{-2G}
    \]
    \item 最大スループット: 
    \[
    S_{\text{max}} = \frac{1}{2e} \approx 0.184 \quad (G = 0.5)
    \]
\end{itemize}

\subsection*{スロット化アロハ方式}
\begin{itemize}
    \item 同期送信: スロット時間に合わせて送信。
    \item スループット公式:
    \[
    S = G e^{-G}
    \]
    \item 最大スループット: 
    \[
    S_{\text{max}} = \frac{1}{e} \approx 0.368 \quad (G = 1.0)
    \]
\end{itemize}

\section{CSMA方式の解析}

\subsection*{nonpersistent CSMA}
\begin{itemize}
    \item キャリアセンス(通信路確認)後、空いていれば即送信。
    \item 使用中の場合、ランダム時間待機後に再試行。
\end{itemize}

\subsection*{スループットの特性}
\begin{itemize}
    \item 正規化伝搬遅延時間:
    \[
    a = \frac{\tau}{T}
    \]
    \item 衝突が減少し、スループットが向上。
\end{itemize}

\subsection*{スループット公式}
\[
S = \frac{G e^{-mG}}{G (1 + 2a) + e^{-mG}}
\]

\newpage

\section{スループット特性の比較}

\begin{table}[h]
    \centering
    \begin{tabular}{@{}lc@{}}
    \toprule
    アクセス方式 & 最大スループット \\ \midrule
    純アロハ方式 & 0.184 \\
    スロット化アロハ方式 & 0.368 \\
    1-persistent CSMA & 0.529 \\
    nonpersistent CSMA & 0.815 \\
    p-persistent CSMA (\( p = 0.03 \)) & 0.827 \\ \bottomrule
    \end{tabular}
    \caption{スループット特性の比較}
\end{table}

\section{平均伝送遅延時間の解析}

\subsection*{純アロハ方式}
\[
D = T + 2\tau + \left(\frac{1}{S} - 1\right) \cdot (T + B)
\]
\begin{itemize}
    \item \( D \): 平均伝送遅延時間。
    \item \( T \): 伝送時間。
    \item \( \tau \): 伝搬遅延時間。
    \item \( B \): 平均バックオフ時間。
\end{itemize}

\end{document}
