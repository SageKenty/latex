\documentclass[a4paper,11pt]{ltjsarticle}
\usepackage[dvipdfmx]{graphicx}
\usepackage{color}
\usepackage{amssymb}
\usepackage{here}
\usepackage{subcaption}
\usepackage{amsmath}
\usepackage{listings}
\usepackage{url}

\lstset{
	%プログラム言語(複数の言語に対応,C,C++も可)
	language = Matlab,
	%背景色と透過度
	backgroundcolor={\color[gray]{.90}},
	%枠外に行った時の自動改行
	breaklines = true,
	%自動改行後のインデント量(デフォルトでは20[pt])	
	breakindent = 10pt,
	%標準の書体
	basicstyle = \ttfamily\scriptsize,
	%コメントの書体
	commentstyle = {\itshape \color[cmyk]{1,0.4,1,0}},
	%関数名等の色の設定
	classoffset = 0,
	%キーワード(int, ifなど)の書体
	keywordstyle = {\bfseries \color[cmyk]{0,1,0,0}},
	%表示する文字の書体
	stringstyle = {\ttfamily \color[rgb]{0,0,1}},
	%枠 "t"は上に線を記載, "T"は上に二重線を記載
 %他オプション:leftline,topline,bottomline,lines,single,shadowbox
	frame = TBrl,
	%frameまでの間隔(行番号とプログラムの間)
	framesep = 5pt,
	%行番号の位置
	numbers = left,
 %行番号の間隔
	stepnumber = 1,
 %行番号の書体
	numberstyle = \tiny,
 %タブの大きさ
	tabsize = 4,
	%キャプションの場所("tb"ならば上下両方に記載)
	captionpos = t
}
\begin{document}

\title{レポート課題(1) \\ 信号処理Ⅱ}

\author{学籍番号:2022531033 氏名:関川謙人}
\date{提出日:\today}
\maketitle

\section{課題1}
重み付き最小自乗法による最適なフィルタ係数の導出式を求めよ。

\begin{equation} \label{eq:1}
  \min_{a} E = \int_{-\pi}^{\pi} |W(\omega)(H(\omega)-D(\omega))|^2
\end{equation}

式(\ref{eq:1})は次式のように行列表現できる
\begin{align}
  E'' &= \sum_{L-1}^{k=0} |W(\omega_{k})(H(\omega_{k})-D(\omega_{k}))^2| \nonumber \\
      &= E^{t}E = (W(Qa - d))^{t}(W(Qa - d)) \nonumber \\
      &= a^{t}Q^{t}(W^{t}W)Qa - 2a^{t}Q^{t}(W^{t} W)d + d^{t}(W^{t}W)d \label{eq:2}
\end{align}

$W = W^{t}$であるため,$E''$は以下のように整理できる
\begin{equation}
  E'' = a^{t}Q^{t}W^{2}Qa - 2a^{t}Q^{t}W^{2}d + d^{t}W^{2}d \label{eq:3}
\end{equation}

式(\ref{eq:2})における、$Q^{t}W^{t}WQ$は常に対称行列である。
評価関数を$a$で偏微分すると、次式が得られる。
\begin{align*}
  \frac{\partial E''}{\partial a} &= 2a^{t}Q^{t} \\
  &= 2Q^{t}W^{2}Qa - 2Q^{t}W^{2}d \\
  &= 2Q^{t}W^{2}(Qa - d) = 0
\end{align*}

これをaについて解くと、
\begin{gather*}
  2Q^{t}W^{2}Qa - 2Q^{t}W^{2}d = 0 \\
  Q^{t}W^{2}Qa = 2Q^{t}W^{2}d \\
  a = (Q^{t}W^{2}Q)^{-1}Q^{t}W^{2}d
\end{gather*}

よって、導出式は、
\begin{equation*}
  a = (Q^{t}W^{2}Q)^{-1}Q^{t}W^{2}d
\end{equation*}

\section{課題2}

\begin{thebibliography}{9}
	\bibitem{Mat_official} MathWorks公式ドキュメント \url{https://jp.mathworks.com/help/matlab/ref/}
	\bibitem{class_ref} 信号処理Ⅱ 2024 講義資料
\end{thebibliography}

\end{document}