\documentclass[a4paper,11pt]{ltjsarticle}


% 数式
\usepackage{amsmath,amsfonts}
\usepackage{bm}
% 画像
\usepackage{graphicx}

\usepackage{caption}
\captionsetup{labelformat=empty,labelsep=none}
\begin{document}

\title{通信工学 演習レポート}
\author{氏名:関川 謙人\\学籍番号:2022531033}
\date{\today}
\maketitle

\section*{問1}
内部抵抗を$R_{m}$、定格電流を$I_{m}$とすると、
\begin{gather*}
  R_{m} = 100 \Omega \\
  I_{m} = 0.5\mathrm{mA} = 5.0 \times 10^{-3} \mathrm{A}
\end{gather*}
1Aの測定範囲においては電流計に$0.5 \times 10^{-3}$Aが流れるため$R_{1}$を通る電圧は$ 0.995$Aである。オームの法則より、電流計に
かかる電圧$V$は
\begin{equation*}
  V = I_{m} \times R_{m} = 0.5 \times 10^{-3} \times 100 = 0.5 \mathrm{V}
\end{equation*}
$R_{1}$と電流計は並列であるため、$R_{1}$にかかる電圧は
\begin{equation*}
  V = 0.995 \mathrm{A} \times R_{1} 
\end{equation*}
である。よって、$R_{1}$は、
\begin{equation*}
  R_{1} = \frac{0.5}{0.995} = 0.5025 \Omega
\end{equation*}
次に10Aの測定範囲を求める。電流計には$0.5 \times 10^{3}$Aが流れるため、
$R_{2}$を通るのは9.995Aである。
電流計の端子電圧は同様に0.5Vであるため、$R_{2}$にかかる電圧は、
\begin{equation*}
  V = 9.995 \times R_{2}
\end{equation*}
よって$R_{2}$は、
\begin{equation*}
  R_{2} = \frac{0.5}{9.995} = 0.0501 \Omega
\end{equation*}
よってこの時、
\begin{gather*}
  R_{1} = 0.5025 \Omega \\
  R_{2} = 0.0501 \Omega
\end{gather*}
\section*{問2}
(1)図2において、平衡条件は
\begin{equation*}
  \frac{Z_{1}}{Z_{2}} = \frac{Z_{3}}{Z_{4}}
\end{equation*}
また、各インピーダンスを以下のように表す。
\begin{gather*}
  Z_{1} = \frac{1}{j\omega C_{1}} \\
  Z_{2} = R_{2} \\
  Z_{3} = \frac{1}{j\omega C _{3}}\\
  Z_{4} = R_{x} + j\omega L_{x}
\end{gather*}
これを平衡条件の式に代入すると、
\begin{equation*}
  \frac{\frac{1}{j\omega C_{1}}}{R_{2}} = 
  \frac{\frac{1}{j\omega C_{3}}}{R_{x}+j\omega L_{x}}
\end{equation*}
これを整理すると、
\begin{equation*}
\frac{1}{j\omega C_{1}R_{2}} = \frac{1}{j\omega C_{3}(R_{x}+j\omega L_{x})}
\end{equation*}
また、実部と虚部を分離し、それぞれ比較すると
\begin{gather*}
  \frac{1}{C_{1}R_{2}} = \frac{R_{x}}{C_{3}(R_{x}^{2}+(\omega L_{x})^{2})} \\
  \frac{\omega L_{x}}{C_{3}(R_{x}^{2} + (\omega L_{x})^{2})} = 0
\end{gather*}
虚部の式より、
\begin{equation*}
  \omega L_{x} = 0
\end{equation*}
これより、$L_{x}$は0である。よって、
\begin{equation*}
  \frac{1}{C_{1}R_{2}} = \frac{1}{C_{3}R_{x}} \\
\end{equation*}
よって平衡条件は、
\begin{equation*}
  R_{x} = \frac{C_{3}}{C_{1}}R_{2}
\end{equation*}
となる。

(2)
\begin{gather*}
  C_{1} = 1\mu\mathrm{F} \\
  C_{3} = 0.5\mu\mathrm{F} \\
  R_{2} = 1\mathrm{k\Omega} = 1000\Omega
  R_{3} = 2\mathrm{k\Omega} = 2000\Omega
\end{gather*}
以上の値を(1)で求めた式に代入すると、
\begin{equation*}
  R_{x} = \frac{0.5}{1} \times 1000 = 500\Omega
\end{equation*}
インダクタンス$L_{x}$は0である。このことから、
\begin{gather*}
  R_{x} = 500\Omega \\
  L_{x} = 0
\end{gather*}
である。
\end{document}