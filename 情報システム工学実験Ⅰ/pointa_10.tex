\documentclass[a4paper,titlepage,11pt]{ltjsarticle}
\usepackage{graphicx}
\usepackage{color}
\usepackage{amssymb}
\usepackage{here}
\usepackage{listings}
\lstset{
	%プログラム言語(複数の言語に対応,C,C++も可)
 	language = C,
 	%背景色と透過度
 	backgroundcolor={\color[gray]{.90}},
 	%枠外に行った時の自動改行
 	breaklines = true,
 	%自動改行後のインデント量(デフォルトでは20[pt])	
 	breakindent = 9pt,
 	%標準の書体
 	basicstyle = \ttfamily\scriptsize\fontsize{8}{9},
 	%コメントの書体
 	commentstyle = {\itshape \color[cmyk]{1,0.4,1,0}},
 	%関数名等の色の設定
 	classoffset = 0,
 	%キーワード(int, ifなど)の書体
 	keywordstyle = {\bfseries \color[cmyk]{0,1,0,0}},
 	%表示する文字の書体
 	stringstyle = {\ttfamily \color[rgb]{0,0,1}},
 	%枠 "t"は上に線を記載, "T"は上に二重線を記載
	%他オプション:leftline,topline,bottomline,lines,single,shadowbox
 	frame = TBrl,
 	%frameまでの間隔(行番号とプログラムの間)
 	framesep = 5pt,
 	%行番号の位置
 	numbers = left,
	%行番号の間隔
 	stepnumber = 1,
	%行番号の書体
 	numberstyle = \tiny,
	%タブの大きさ
 	tabsize = 4,
 	%キャプションの場所("tb"ならば上下両方に記載)
 	captionpos = t
}
\begin{document}
\begin{lstlisting}[caption=enshukadai9]
  void warshall_floyd()
  {
      int i ,j ,k;
  
      for(i = 1; i <= n; j++)
      {
          for(j = 1; j <= n ; j++)
          {
              d[i][j] = w[i][j]; //Weight of d[i][j] = w[i][j]
          }
      }
      for(k = 1; i <= n; k++)
      {
          for(i = 1;i <= n; i++;)
          {
              for(j = 1; j <= n; j++)
              {
                  d[i][j] = min(d[i][j],d[i][k]+d[k][j]);//d^k-1(i,j)=d^k(i,j)
              }
          }
      }
  }
\end{lstlisting}
\end{document}